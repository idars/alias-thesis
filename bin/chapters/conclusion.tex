\chapter{Conclusion}
\label{ch:conclusion}

The conclusion is based on several factors, including the evaluation of the final prototype, the research question and the aims that were set. Taking these factors into consideration, this project is found to be successful.

\section{Discussion}
\label{sec:discussion}

% Arguments for why the conclusion is as such, what went right and what went wrong

The biggest factor is the evaluation of the final prototype. As this evaluation was mainly positive, 



The aims that were set in \autoref{sec:researchquestion} have been accomplished. 

The means of the evaluation could have been better, most notably the target group which has little resemblance to the actual target group. As previously stated (\autoref{sec:testgroup}), this was due to several risks combined with the prototype. However, as the users are pretty renown in their fields, the evaluation has a certain value. Looking back, a possible improvement would be to evaluate the prototype with a group of non-hospitalized children who are aware of the difference between an unfinished prototype and a finished product.

It would also have been preferred if a prototype built on the intended architecture was made. Only code-less prototypes were made, focusing on the design rather than the implementation. Time was the biggest factor as to why a software implementation was only planned but not attempted. This part has not been evaluated in any form and therefore is not a factor to the conclusion.

\section{Reflections}
\label{sec:reflections}

% Things to consider





\section{Further work}
\label{sec:furtherwork}

(...)

\subsection{Design and planning}

(...)
% 'Merge' guideline pathways

\subsection{Development}

(...)
