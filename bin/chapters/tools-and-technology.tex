\chapter{Tools and technology}
\label{ch:tools}

\section{Initial remarks}

Requirements were not really present for this project; instead, there are various interests in how the application should be made and what it should result in.

To begin with, the Children and Youth Clinic indicated that this was intended for tablets with medium to large screens. Such tablets usually run a full-fledged operating system (OS) such as Windows or a mobile OS such as Android and iOS. There are no requirements regarding which operating systems the software should run on, but it seems that most of their tablets run Windows and Android operating systems.

As with most mobile applications, there is a choice between the following approaches:

\begin{itemize}
    \item A \emph{native application} is written in each operating system's native languages. For Android this is Java, and Swift and Objective-C for iOS. Native applications can access all features which each OS may offer.
    \item Another native approach is to write the application in a different language and compile it to native code. The resulting app is then very alike a native one and has similar performance.
    \item A \emph{hybrid application} encapsulates a web page into an app. Such apps use a browser instance to render elements, although without the search bar and tools of the browser. This approach offers functionality that you don't get with web applications, but has usually worse performance compared to a native solution.
    \item A mobile \emph{web application} is a responsive web page shown in the user's web browser. These do not appear in app stores but function just like an ordinary web page. Lately there have been increasing interest in Progressive Web Apps (PWAs) which aim to provide mobile web pages with app-like behaviour and functionality such as push-notifications.
\end{itemize}

Due to the uncertainty in which operating systems that are in use (and will be used in the future), going for a cross-platform application is the most preferred. 

Flutter is a relatively new framework

Dersom en Windows-løsning er ønskelig vil dette muligens kreve kunnskap om .NET, noe som ikke er blitt lært hittil. Det er flere måter å utvikle Windows 10-eksklusive applikasjoner på, men jeg har ikke gått i detalj gjennom disse ennå.

Dersom vi lander på å utvikle for mobile operativsystemer vil det være naturlig å lage en multiplatform-applikasjon. Dersom ønskelig kan man etterpå se på en web-applikasjon eller en Windows desktop app.

- Xamarin (Microsoft)
	- C og F
	- Stabilt og robust rammeverk
	- Kan utvikle for Windows 10 samtidig som for Android og iOS
	- Dårligere ytelse
	- Tidkrevende
- Flutter (Google)
	- Dart (C-lignende syntaks, kompilerer til JavaScript)
	- God ytelse
	- Hot reload
	- Ganske nytt; mangler erfaring som andre rammeverk har
- React Native (Facebook)
	- HTML/CSS/JavaScript
	- Bruker native UI-elementer
	- Hot reload
	- Kan gjenbruke kode for eventuell nettside
	- Dårligere ytelse
	- Relativt vanskelig å teste
- PhoneGap (Adobe) / Cordova (Apache)
    - HTML/CSS/JavaScript
    - Tradisjonelle web-rammeverk
    - Delvis manglende støtte for native funksjonalitet
- Titanium
    - JavaScript
    - 

Flutter har stort potensiale for utvikling av mobilapplikasjoner og ser ut til å passe oppgavebeskrivelsen på en god måte. Dersom man ønsker å utvikle en og samme app både for Windows 10 og for Android/iOS vil Xamarin være et lovende alternativ.

Angående 3D-rotering av 2D-bilder kan dette oppnås enten ved å bruke en rekke bilder som er ferdig rotert, eller å faktisk rotere 2D-bilder i 3D. Xamarin har støtte for sistnevnte via SkiaSharp. Flutter støtter 3D-rotering også. Det ser ut til at React Native ikke har native støtte for dette, men det finnes trolig JavaScript-plugins som kan tilføye slik funksjonalitet.

\section{Anticipated challenges and feasibility}

The development tools chosen for an application should support the functionality of the application. The following subsections illustrate a few scenarios which the chosen development tools should support.

\subsection{Projecting the avatar on the screen}

Given that the avatar generation system only outputs images seen from the front view angle only, it will require additional work to make these look natural in a three-dimensional space. This has also been taken into consideration when deciding which tool to use for developing the application.

The avatar generation system created for E-LAN (from \ref{sec:relatedwork}) can be used together with the application. This enables the user to view their avatar in procedures like they were participating themselves. The system does, however, only output portrait pictures in 2D and concerns were raised about whether these would look realistic in certain settings. For example, using a single 2D image, a person laying in the bed would look awkward unless viewed from above the bed. There are in fact multiple approaches to this as seen in table \ref{tab:projecting-avatar}

% Figure with 4 images showing the different projections?

\begin{table}
    \centering
    \begin{tabu}{L[0.75] l L[0.5] l L}
        \textbf{} & \textbf{Realism} & \textbf{Processing \newline power} & \textbf{Ease of use} & \textbf{Additional \newline requirements} \\ \hline
        \textbf{2D images} & Lowest & Lowest & Highest & None \\ \tabucline[hdottedline]{-}
        \textbf{2D image sets with various poses} & High & Low & High & Extra image sets \\ \tabucline[hdottedline]{-}
        \textbf{2D images rotated in 3D} & Low & High & Low & Software framework which supports 3D rotations \\ \tabucline[hdottedline]{-}
        \textbf{3D models} & Highest & Highest & Lowest & New 3D models; a 3D rendering engine; software framework which supports 3D rotations \\ \hline
    \end{tabu}
    \caption{Different ways to project an avatar on a screen}
    \label{tab:projecting-avatar}
\end{table}

It is shown that 2D images can be rotated in 3D pretty realistically. \textcite{rivers2010} carried out a project which showed that it is possible to view a figure from any angle when given three 2D projections of it.

Though, an alternative is to simply use such avatars in 2D-space. 


The IACTA application shows that this can be used with similar effect as a 3D-application \parencite{stalberg2018}.

\section{A future handoff}

The scope of this project involves minimal integration with existing healthcare and journal systems at Barneklinikken and Haukeland. Given that Helse Vest IKT monitors most of said systems, it would be sensible to develop an application that can be adapted, or even be developed further on, by them. It was pointed out that the developers of the avatar generation system used well-established web technologies such as HTML, CSS and JavaScript to develop it, and that similar technologies were preferred for the new application. This led to a new direction in choosing the most suitable software tools.

With this in mind, there were two main approaches remaining:

The first approach involves a mobile app and can be achieved in two ways; one way is to compile an application into native code. There are a number of frameworks which are able to transpile JavaScript into native applications, although the syntaxes may vary. Another way is to use \emph{hybrid applications}, using HTML, CSS and JacaScript. These are typically built upon using a WebView, a browser instance that can be used by the application. Some frameworks offer their own JS-like language, providing additional methods and native functionality, which are compiled into traditional JavaScript.

The remaining approach is through web applications which run in web browsers. Lately there have been increasing interest in \emph{Progressive Web Apps} (PWAs), which aim to make websites more app-like on mobile devices with offline access and push notifications. They do, however, require a website which is capable of serving HTTPS. PWAs is a relatively new kind of technology with increasing support in both Android and iOS. Some browsers also support adding an app icon to the user's app launcher, circumventing app stores in the process.

\section{Aiming for an approach}

Determining which approach to use requires paying attention to several factors: requirements, anticipated challenges and stakeholder interests among others. The requirements suggest using a cross-platform application primarily aimed for tablet devices, while Helse Vest IKT suggest using web technologies when developing the application. General factors such as cost, ease of use, responsiveness, support for older devices and debugging also play a role.

When taking Helse Vest IKT's opinion into account, there are basically two types of applications left to choose from. 

Although only 2D images are currently supported, it seems reasonable to pick an approach that allow more extensive functionality if desired.

Given that both hybrid applications and Progressive Web Applications use web browsers (WebViews) to show content on the screen, their performances are assumed to be pretty comparable. The main difference is how a hybrid application is dependent on an app store whereas a PWA is dependent on a website. In this case, a self-running application will require less server resources

Based on the reflections above, a Progressive Web Application is considered to be less suitable for this project.

Next to consider is frameworks. Frameworks allow developers to develop an application more efficiently by facilitating APIs, UI components, navigation, MVC patterns, utility methods or a combination of these. Some frameworks can also help deploying the application to app stores.

Table \ref{tab:frameworks} shows a few frameworks and what functionalities they offer.

\begin{table}
    \centering
    \begin{tabu}{l l L}
        \textbf{Framework} & \textbf{Functionality} & \textbf{Description} \\
        \hline
        Meteor             & Native compiler    & JavaScript \\
        NativeScript       & Native compiler    & JavaScript \\
        React Native       & Native compiler    & Mobile version of React. Supports hot reloading. Focuses on Single Page Applications (SPAs) \\
        \tabucline[hdottedline]{-}
        Apache Cordova     & App wrapper & \\
        Adobe PhoneGap     & App wrapper & \\
        Ionic Capacitor    & App wrapper & \\
        \tabucline[hdottedline]{-}
        Angular            & Navigation       & \\
        Framework7         & Navigation \& UI & \\
        Ionic              & Navigation \& UI & \\
        React (+ Flux)     & Navigation \& UI & JavaScript and JSX (HTML-like syntax represented as JavaScript objects). Supports hot reloading \\
        Vue.js             & Navigation \& UI & \\
        \tabucline[hdottedline]{-}
        Bootstrap          & UI               & \\
        Onsen              & UI               & \\
        Polymer            & UI               & \\
        Semantic UI        & UI               & \\
        \hline
    \end{tabu}
    \caption{Web developer friendly frameworks for mobile application development}
    \label{tab:frameworks}
\end{table}

The most common way to make a hybrid app is to use a Cordova-application. Apache Cordova is the original concept Adobe PhoneGap is an extension to Cordova and acts much the same with some additional features.

It is unknown which frameworks the web developers of Helse Vest IKT have used previously, if any. A starting point is to assume that web developers have no or poor prior experience with frameworks. Any framework used for the project should therefore be easy to learn for any person skilled in HTML, CSS and JavaScript.

A framework worth explaining is React
React Native is a mobile version of React, able to transpile code to both Android and iOS.

Is it too difficult for a web developer to learn React and its adjacent technologies? Learning React most likely requires learning JSX, Redux and Flux as well. Although 

Supported functionalities
What functionality do the frameworks facilitate?

\section{Database system}

SQL vs NoSQL

\section{Technologies used}

\subsection{Framework}


CSS-in-JS - makes it easier to utilize visual themes

\subsection{JSON}

Lagringsmessig vil historier være serialiserte og kunne lagres i databaser. For å overføre historier kan de sendes i form av JSON-strenger som vil beskrive informasjon som

- Paneler
	- Antall paneler, evt. navn på disse
	- Størrelse på panel(er)
- Piktogram/avatar
    - Plassering
	- Rotasjon
	- Størrelse
	- Om piktogrammet/avataren er speilvendt
	- Farge
- Bakgrunn
    - Farge eller bilde

JSON kan også benyttes til å overføre kontoinformasjon, innstillinger og lignende. Bilder og piktogrammer overføres som filer.
