\chapter{Tools and technology}
\label{ch:tools}

\section{Requirements}

The Children and Youth Clinic expects an application where the user can view personally targeted procedures. These will feature the user's own personal avatar along with information about an upcoming procedure at the hospital. Afterwards, the user should be able to rate their experience, and if possible, this rating should be reflected when the procedure is shown in retrospect.

The target group will be children and youth at the clinic with ages raging from 5 to 12. The content of the application must therefore be adapted to the target group and be suitable for their age. An essential plan when it comes to the design of the application is to let children of the intended age group test it in various stages of its development. Their input is valuable since it can contribute to making the application age-appropriate \parencite{stalberg2016}.

The clinic expressed that they intend to use the application on larger screens and most likely on tablets.

\section{Choice of application setup}

Requirements were not really present for this project; instead, there are various interests in how the application should be made and what it should result in.

To begin with, the Children and Youth Clinic indicated that this was intended for tablets with medium to large screens. Such tablets usually run a full-fledged operating system (OS) such as Windows or a mobile OS such as Android and iOS. There are no requirements regarding which operating systems the software should run on, but it seems that most of their tablets run Windows and Android operating systems.

As with most mobile applications, there is a choice between the following approaches:

\begin{itemize}
    \item A \emph{native application} is written in each operating system's native languages. For Android this is Java, and Swift and Objective-C for iOS. Native applications can access all features which each OS may offer.
    \item Another native approach is to write the application in a different language and compile it to native code. The resulting app is then very alike a native one and has similar performance.
    \item A \emph{hybrid application} encapsulates a web page into an app. Such apps use a browser instance to render elements, although without the search bar and tools of the browser. This approach offers functionality that you don't get with web applications, but has usually worse performance compared to a native solution.
    \item A mobile \emph{web application} is a responsive web page shown in the user's web browser. These do not appear in app stores but function just like an ordinary web page. Lately there have been increasing interest in Progressive Web Apps (PWAs) which aim to provide mobile web pages with app-like behaviour and functionality such as push-notifications.
\end{itemize}

Due to the uncertainty in which operating systems that are in use (and will be used in the future), going for a cross-platform application is the most preferred. 

Flutter is a relatively new framework

Dersom en Windows-løsning er ønskelig vil dette muligens kreve kunnskap om .NET, noe som ikke er blitt lært hittil. Det er flere måter å utvikle Windows 10-eksklusive applikasjoner på, men jeg har ikke gått i detalj gjennom disse ennå.

Dersom vi lander på å utvikle for mobile operativsystemer vil det være naturlig å lage en multiplatform-applikasjon. Dersom ønskelig kan man etterpå se på en web-applikasjon eller en Windows desktop app.

- Xamarin (Microsoft)
	- C og F
	- Stabilt og robust rammeverk
	- Kan utvikle for Windows 10 samtidig som for Android og iOS
	- Dårligere ytelse
	- Tidkrevende
- Flutter (Google)
	- Dart (C-lignende syntaks, kompilerer til JavaScript)
	- God ytelse
	- Hot reload
	- Ganske nytt; mangler erfaring som andre rammeverk har
- React Native (Facebook)
	- HTML/CSS/JavaScript
	- Bruker native UI-elementer
	- Hot reload
	- Kan gjenbruke kode for eventuell nettside
	- Dårligere ytelse
	- Relativt vanskelig å teste
- PhoneGap (Adobe) / Cordova (Apache)
    - HTML/CSS/JavaScript
    - Tradisjonelle web-rammeverk
    - Delvis manglende støtte for native funksjonalitet
- Titanium
    - JavaScript
    - 

Flutter har stort potensiale for utvikling av mobilapplikasjoner og ser ut til å passe oppgavebeskrivelsen på en god måte. Dersom man ønsker å utvikle en og samme app både for Windows 10 og for Android/iOS vil Xamarin være et lovende alternativ.

Angående 3D-rotering av 2D-bilder kan dette oppnås enten ved å bruke en rekke bilder som er ferdig rotert, eller å faktisk rotere 2D-bilder i 3D. Xamarin har støtte for sistnevnte via SkiaSharp. Flutter støtter 3D-rotering også. Det ser ut til at React Native ikke har native støtte for dette, men det finnes trolig JavaScript-plugins som kan tilføye slik funksjonalitet.

\section{Aiming for an approach}

Determining which approach to use requires paying attention to several factors: requirements, anticipated challenges and stakeholder interests among others. The requirements suggest using a cross-platform application primarily aimed for tablet devices, while Helse Vest IKT suggest using web technologies when developing the application. General factors such as cost, ease of use, responsiveness, support for older devices and debugging also play a role.

When taking Helse Vest IKT's opinion into account, there are basically two types of applications left to choose from. 

Although only 2D images are currently supported, it seems reasonable to pick an approach that allow more extensive functionality if desired.

Given that both hybrid applications and Progressive Web Applications use web browsers (WebViews) to show content on the screen, their performances are assumed to be pretty comparable. The main difference is how a hybrid application is dependent on an app store whereas a PWA is dependent on a website. In this case, a self-running application will require less server resources

Based on the reflections above, a Progressive Web Application is considered to be less suitable for this project.

Next to consider is frameworks. Frameworks allow developers to develop an application more efficiently by facilitating APIs, UI components, navigation, MVC patterns, utility methods or a combination of these. Some frameworks can also help deploying the application to app stores.

Table \ref{tab:frameworks} shows a few frameworks and what functionalities they offer.

\begin{table}[h]
    \centering
    \begin{tabu}{l l L}
        \textbf{Framework} & \textbf{Functionality} & \textbf{Description} \\
        \hline
        Meteor             & Native compiler    & JavaScript \\
        NativeScript       & Native compiler    & JavaScript \\
        React Native       & Native compiler    & Mobile version of React. Supports hot reloading. Focuses on Single Page Applications (SPAs) \\
        \tabucline[hdottedline]{-}
        Apache Cordova     & App wrapper & \\
        Adobe PhoneGap     & App wrapper & \\
        Ionic Capacitor    & App wrapper & \\
        \tabucline[hdottedline]{-}
        Angular            & Navigation       & \\
        Framework7         & Navigation \& UI & \\
        Ionic              & Navigation \& UI & \\
        React (+ Flux)     & Navigation \& UI & JavaScript and JSX (HTML-like syntax represented as JavaScript objects). Supports hot reloading \\
        Vue.js             & Navigation \& UI & \\
        \tabucline[hdottedline]{-}
        Bootstrap          & UI               & \\
        Onsen              & UI               & \\
        Polymer            & UI               & \\
        Semantic UI        & UI               & \\
        \hline
    \end{tabu}
    \caption{Web developer friendly frameworks for mobile application development}
    \label{tab:frameworks}
\end{table}

The most common way to make a hybrid app is to use a Cordova-application. Apache Cordova is the original concept Adobe PhoneGap is an extension to Cordova and acts much the same with some additional features.

It is unknown which frameworks the web developers of Helse Vest IKT have used previously, if any. A starting point is to assume that web developers have no or poor prior experience with frameworks. Any framework used for the project should therefore be easy to learn for any person skilled in HTML, CSS and JavaScript.

A framework worth explaining is React
React Native is a mobile version of React, able to transpile code to both Android and iOS.

Is it too difficult for a web developer to learn React and its adjacent technologies? Learning React most likely requires learning JSX, Redux and Flux as well. Although 

Supported functionalities
What functionality do the frameworks facilitate?

\section{Database system}

SQL vs NoSQL

\section{Technologies used}

\subsection{Framework}


CSS-in-JS - makes it easier to utilize visual themes

\subsection{JSON}

Lagringsmessig vil historier være serialiserte og kunne lagres i databaser. For å overføre historier kan de sendes i form av JSON-strenger som vil beskrive informasjon som

- Paneler
	- Antall paneler, evt. navn på disse
	- Størrelse på panel(er)
- Piktogram/avatar
    - Plassering
	- Rotasjon
	- Størrelse
	- Om piktogrammet/avataren er speilvendt
	- Farge
- Bakgrunn
    - Farge eller bilde

JSON kan også benyttes til å overføre kontoinformasjon, innstillinger og lignende. Bilder og piktogrammer overføres som filer.
