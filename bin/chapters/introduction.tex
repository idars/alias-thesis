\chapter{Introduction}
\label{ch:introduction}

\section{About the thesis}
\label{sec:about}

This thesis describes an application development project that was held from {\printdayoff \daterange{2018-08-00}{2019-12-00}}. The project supervisor, Carsten Helgesen, proposed the project as a continuation of a bachelor project from the spring of 2018. Both projects seek to improve communication and engagement by facilitating non-verbal communication and to visualise information and emotions.

Around middle of 2018, the project gained interest at the Children and Youth Clinic at Haukeland University Hospital. Ideas for the project have been discussed since then with senior consultant psychiatrist Paul Joachim Bloch Thorsen.

\section{Motivation}
\label{sec:motivation}

Communication between patients and medical staff is crucial in healthcare. It is important for the ones who get treated to know what they get treated for why the treatment is as such. In addition, the ones who give the treatment can benefit from knowing what the patients feel about the treatments. However, this flow of communication is not always facilitated.

% Hvorfor, hvilken nytte det vil ha

% Problem description

Hospitals carry a lot of information about various treatments and procedures, and often does a nice job putting this information out to patients. The problem, though, is that this information is often not personalised and motivation from patients is often lacking. The amount of information can sometimes also be overwhelming, putting the patients at risk of overlooking important details. This again may lead to missed appointments, unnecessarily long hospital stays and increased costs for hospitals.

% Mange informasjonsbrosjyrer, ikke tilpasset

For younger patients, these issues become even greater. Although national guideline pathways (see \autoref{sec:pathways}) aim to make the treatments more efficient for everyone, the information about them is still mostly text-based and not properly adapted to this age group, contributing to the fact that children are rarely involved in decision-making \autocite{coyne2008}. Though, technology that can be tailored towards children and youth may be beneficial:

\begin{quote}
    Age-appropriate interactive technology can be used to promote young children’s understanding and to facilitate their situated participation in healthcare situations.

    \raggedleft--- \textcite{stalberg2018}
\end{quote}

There is also room for improvement in patient-to-healthcare communication. It is important for patients to give feedback about one's treatment, but this is not always easy. The feedback is often restricted to paper sheets and verbal communication which can result in loss of data and extra work for the medical staff.

\section{Research question and expected results}
\label{sec:researchquestion}

When forming the research question, the idea is to see how these problems can be approached. Many of them are a consequence of poor communication from hospital to patients and vice versa. Personalizing the information given from hospitals is one way to improve the communication that can result in several positive benefits; increased motivation, increased quality of life and increased participation in decision-making being a few of them. By focusing on personalization of national guideline pathways, the project may contribute towards several of these problems while working with a local, less extensive domain area.

Therefore, the research question to be asked is:

\begin{quote}
    How can national guideline pathways be made more personalised for children and youth?
\end{quote}

Following the research question, two aims are set up for this project to achieve. These should contribute to make the solution beneficial for younger patients:

\begin{itemize}
    \item The first aim is to look at how the information from hospitals may be formed and the way it is given, and adjust this information flow such that it is more engaging and understandable for children and youth.
    \item The second aim is to facilitate how children and youth may participate in decision-making and give feedback about their treatment as a step in personalizing the treatment itself.
\end{itemize}

At project completion, the desired result is a functional prototype (see \autoref{sec:prototyping}). This prototype should be extensive and functional enough to be developed further on by a healthcare-centralised software development company, or possibly by students as part of another thesis. Given the positive outcome of E-LAN (see \autoref{sec:relatedwork}), it is a possibility that the application may be used among children at the clinic sometime in the future.

% Si mer om hva en funksjonell prototype er
% Kan bruke figur, x=funksjonalitet og y=visuell

As the Children and Youth Clinic has planned a bigger project involving the use of avatars and comics, the outcome of this project will become an indication of whether it is valuable to invest in it. This project will also provide useful knowledge that may come in handy for an eventual succeeding project.

\section{Thesis outline}
\label{sec:outline}

The thesis is structured as follows: \Cref{ch:background} lists relevant topics of interest and problem areas while giving an insight into preceding projects. In \autoref{ch:methodology}, the project's methodology and approach to the problem is described. Following that, the design process is planned in \autoref{ch:planning} and then detailed out as it is performed in \autoref{ch:iterating}. The result of this design process is then evaluated in \autoref{ch:evaluation}. \Cref{ch:outline} outlines the thought application and its challenges while \autoref{ch:tools} discusses the setup, architecture and tools needed to make said application. Finally, \autoref{ch:conclusion} concludes the project and discusses further work.
