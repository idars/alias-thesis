\chapter{Introduction}
\label{ch:introduction}

\section{About the thesis}

This thesis describes an application development project that was held from August 2018 to June 2019. Carsten Helgesen, the project supervisor, proposed the project as a continuation of a bachelor project from the spring of 2018. Both projects aim to facilitate non-verbal communication and to visualise information and emotions through human figures in context. The original intention was to create a communication tool aimed for people with autism, but the project has taken a new direction since then.

Throughout the second and third quarters of 2018, the project gained interest at the Children and Youth Clinic (CYC) at Haukeland University Hospital. Ideas for the project were discussed during this period with Consultant (Child and Youth Psychiatrist) Paul Joachim Thorsen.

More to come as the project endures.

\section{Problem description}

Children situated at hospitals

It is a recurring problem that 

\begin{quote}
    Age-appropriate interactive technology can be used to promote young children’s understanding and to facilitate their situated participation in healthcare situations \cite{stalberg2018}
\end{quote}

\section{Research question and expected results}

Research question?

\begin{quote}
    Can children at Haukeland be informed about upcoming procedures in a more informational and engaging way?
    
    How can children at Haukeland be informed about upcoming procedures in a more informational and engaging way?
    
    How can personally catered cartoons engage children and youth at hospitals?
\end{quote}

At project completion, the result will be a functional prototype that can be developed further on by Helse Vest IKT. Given the positive outcome of E-LAN (see \ref{sec:relatedwork}), it is anticipated that the application will be used among children at the clinic.

Since the clinic has planned a bigger project involving the use of avatars and comics, the outcome of this project will become an indication of whether it is valuable to invest in it. This project will also provide useful knowledge that may come in handy for an eventual succeeding project.

\section{Thesis outline}

The thesis is structured as follows: Chapter \ref{ch:background} describes relevant terminology and the theoretical foundation while giving an insight into preceding projects. The development process is detailed out in its entirety in Chapter \ref{ch:design}. This development is then analysed and evaluated in Chapter \ref{ch:analysis} while Chapter \ref{ch:conclusion} concludes the project. Finally, a discussion about further work is made in Chapter \ref{ch:furtherwork}.
