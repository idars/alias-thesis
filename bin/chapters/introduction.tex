\chapter{Introduction}
\label{ch:introduction}

\section{About the thesis}

This thesis describes an application development project that was held from August 2018 to December 2019. The project supervisor, Carsten Helgesen, proposed the project as a continuation of a bachelor project from the spring of 2018. Both of these projects aim to facilitate non-verbal communication and to visualise information and emotions.

Around middle of 2018, the project gained interest at the Children and Youth Clinic at Haukeland University Hospital. Ideas for the project have been discussed since then with senior consultant psychiatrist Paul Joachim Bloch Thorsen.

\section{Motivation}

Hvorfor, hvilken nytte det vil ha

\section{Problem description}

Information that is handed out at most hospitals 

Mange informasjonsbrosjyrer, ikke tilpasset

\begin{quote}
    Age-appropriate interactive technology can be used to promote young children’s understanding and to facilitate their situated participation in healthcare situations \cite{stalberg2018}
\end{quote}

\section{Research question and expected results}

Research question:

\begin{quote}
    How can national guidelines - pakkeforløp - be made more personalised towards children and youth at hospitals?
\end{quote}

The problem area was presented first and foremost by the Children and Youth Clinic. During meetings, Thorsen stated that they wanted to improve the ways of which children were informed about upcoming procedures. Currently, the information that is given here is primarily textual and of varying interest for younger patients.

At project completion, the desired result is a functional prototype. This prototype should be extensive and functional enough to be developed further on by a healthcare-centralized software development company such as Helse Vest IKT or Avans, or eventually by students as part of another thesis. Given the positive outcome of E-LAN (see \cref{sec:relatedwork}), it is anticipated that the application will be used among children at the clinic.

% Si mer om hva en funksjonell prototype er
% Kan bruke figur, x=funksjonalitet og y=visuell

Since the clinic has planned a bigger project involving the use of avatars and comics, the outcome of this project will become an indication of whether it is valuable to invest in it. This project will also provide useful knowledge that may come in handy for an eventual succeeding project.

\section{Thesis outline}

The thesis is structured as follows: \cref{ch:background} lists relevant topics of interest and problem areas while giving an insight into preceding projects. In \cref{ch:methodology}, the project's methodology and approach to the problem is described. Following that, the design process is planned in \cref{ch:planning} and then detailed out as it is performed in \cref{ch:iterating}. The result of this design process is then evaluated in \cref{ch:evaluation}. \Cref{ch:outline} outlines the thought application and its challenges while \cref{ch:tools} discusses the setup, architecture and tools needed to make said application. Finally, \cref{ch:conclusion} concludes the project and discusses further work.
