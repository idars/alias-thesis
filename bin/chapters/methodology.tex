\chapter{Methodology}
\label{ch:methodology}

This project functions as a pilot study in preparation for a bigger project held at the clinic. It is also a comparative study as it may possibly replace the current way of informing patients. This allows the clinic to run a small-scale project and see how the application compares to the existing system at an early stage with reduced investment and costs.

The development will focus on iterating over designs and prototypes in a user-centered manner. Users, both employees and children at the clinic, will be able to try out the design throughout various phases of its development. This user testing may consist of focus groups and uncontrolled experiments, and the gained experience can be applied in the next development stage. The testing will most likely be restricted to the internal group at first, but a designated test group may be created once the design evolves into prototypes. Elements of Design Thinking might also be considered.

\section{Evaluation (plan)}

The final prototype will be evaluated by comparing it with the current system. A group of two to six users of the intended age group will be invited to test and evaluate the application while a control group of the same size will test the current system under the same conditions. The users will quantitatively rate the systems by giving scores from one to five in areas such as "fun", "understandable" and "interesting". The project is deemed to be valuable if the users find the application to be more informative and engaging than the current system.

It is yet to be decided if the test groups will consist of children situated at the clinic, i.e. the target group, or children around the same age.

\section{Design process}

Iterative

For hver iterasjon
- Hva er problemstillingen?
- Hvilken tilnærming til løsningen skal prototypen ha?
- Hvordan er det levert?
- Hvordan det er testet/testresultater

- Double Diamond approach: solve the right problem

https://www.designcouncil.org.uk/news-opinion/design-process-what-double-Diamond

Må utforske mye før man fokuserer inn på problemstilling - utforskningen er iterativ

\section{Interaction design}



\section{Prototyping}

\subsection{Prototyping tools}

