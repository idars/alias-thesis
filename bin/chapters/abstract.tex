\begin{abstract}
    Information that is handed out by hospitals is not always personalised for patients and their motivation is often lacking. Communication between healthcare and patients is mostly textual and verbal, sometimes overwhelming, leaving a risk of missing out on details. For children and youth, this can become a significant issue. This project acts as a pilot study and explores different approaches to personalizing this information such that it becomes more engaging and understandable for younger patients. An iterative design process contributes to shaping the design in order to meet requirements and aims. The design builds upon a national guideline pathway which has recently been incorporated across Norway. The evaluation shows that the final design is easy to use, informative and self explanatory. An outline for a future implementation of the application is then described and discussed.
\end{abstract}
