\chapter{Background}
\label{ch:background}

\section{Preceding projects}

This project builds upon experience from a bachelor thesis named \emph{PictogramApp} which was based on another project named \emph{Pictogram-me}.

\subsection{Pictogram-me}

Since 2011, associate professor in graphic design Linda Lien and professor in visual communication Ashley Booth have researched on creative usage of pictograms. A pictogram, also called a pictograph, is a simplified figure that resembles and represents a physical object. They vary in shapes and sizes, but they are ultimately designed in a way that make them easy to interpret and understand their symbolic meaning.

Lien's and Booth's artistic research project, named \emph{Pictogram-me}, experiments how pictograms can be used to express complex social messages \parencite{lien2018}. The aim is to illustrate challenging situations that people who have a difficult life may endure. Despite pictograms being flat and simplified, Lien and Booth wanted to show how pictograms also can visualise difficult topics and promote empathy.

Pictogram-me presents a new set of pictograms that are designed for the purpose of the project. In addition, the project has resulted in various concepts including

\begin{itemize}
    \item \emph{PictoBooth}, a photo booth that translates the body and gestures into real life pictograms,
    \item \emph{PictoFont}, a symbol typeface consisting of various pictograms, and
    \item \emph{PictoTheatre}, a small-scale theatre where pictograms can be arranged on a scene. A tablet can be placed behind the scene and function as a background as illustrated in \ref{fig:pictotheatre}.
\end{itemize}

\begin{figure}
    \centering
    \includegraphics[width=0.55\textwidth]{pictotheatre.jpg}
    \caption{PictoTheatre, shown at the 2016 RØST conference in Bergen}
    \label{fig:pictotheatre}
\end{figure}

\subsection{PictogramApp}

In 2017, the Western Norway University of Applied Sciences issued out a bachelor project in collaboration with Linda and Booth, with the purpose of creating a smartphone application. The application, which was later named \emph{PictogramApp}, was meant to be a digital version of PictoTheatre where pictograms can be arranged on the screen and form visual messages in a mobile manner \parencite{fure2017}. The application allows users to place pictograms in context in order to create their own stories -- see figure \ref{fig:pictogramapp}. PictogramApp was targeted towards the Church City Mission, a voluntary organisation which offers help and services for people living near the street. A functional prototype of the application was released in June 2017.

\begin{figure}
    \centering
    \begin{subfigure}{0.3\textwidth}
        \centering
        \includegraphics[width=\textwidth]{pictogramapp-01.png}
        \subcaption{List of stories}
        \label{fig:pictogramapp-list}
    \end{subfigure}
    \hspace{0.05\textwidth}
    \begin{subfigure}{0.3\textwidth}
        \centering
        \includegraphics[width=\textwidth]{pictogramapp-02.png}
        \subcaption{Scene editing}
        \label{fig:pictogramapp-scene}
    \end{subfigure}
    \caption{Screenshots from PictogramApp}
    \label{fig:pictogramapp}
\end{figure}

\section{Related work}
\label{sec:relatedwork}

The Children and Youth Clinic has prior to this project experimented with different ways to engage their patients. Among these was an e-sport event named \emph{E-LAN}, held in the end of October 2018. The purpose of this event was to connect gaming towards a healthy lifestyle and to let children and youth master various areas of interest. As a part of this initiative, an avatar generation system was created that let users create personal avatars which represent themselves. Each user would then carry their avatar in a name tag attached on their clothing. The software seems to run on Windows with support for a web client, and outputs two-dimensional portrait pictures.

Several applications and prototypes have been made that aim to provide information about and illustrate a child's hospital stay. A notable example is \emph{IACTA}, short for \emph{Inter-Active Communication Tool for Activities}. This application ()...) \parencite{stalberg2018}.

Another example is an inpatient portal application named \emph{MyChart Bedside}, developed by Epic Systems Corporation for tablet devices. A study conducted by \textcite{kelly2017} revealed that 90 percent of children's parents were satisfied with the portal. 

Bitmoji

Instruction videos used by Norwegian on their airplanes

\section{Hospital settings and terminology}
\label{sec:medicterminology}

Given that the application will be used in a hospital setting, the associated terminology will be extended to the application. A \emph{procedure} is a sequence of steps separated from each other. Each \emph{step} contains a background, an avatar and elements related to the procedure in the form of illustrations or pictures, and may feature interactive as well as non-interactive elements.

Story - scene
Comic strip - panel
Procedure - step

\section{Litterature background}

A number of queries were performed on academic literature search engines in order to discover related work and gain further insight in the problem area. Each query contained a set of the following keywords:

\begin{multicols}{3}
    \raggedcolumns
    \begin{itemize}
        \item Hospital
        \item Patient
        \item Pediatric
        \item Children
        \item Information
        \item Informative
        \item Interactive
        \item Understanding
        \item Comprehension
        \item Engage
        \item Cartoon
        \item Comics
        \item Illustrations
        \item Personalised
    \end{itemize}
\end{multicols}

% List relevant papers that cover the problem area?
